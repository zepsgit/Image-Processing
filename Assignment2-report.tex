\documentclass[14pt]{article}
\title{Lab1 Report}
\author{Zepeng Chen}
\date{\today}
\usepackage{listings}
\usepackage{color}
\usepackage{graphicx}
\usepackage{subcaption}
\usepackage{gensymb}
\usepackage{geometry}
\geometry{a4paper,left=2cm,right=2cm,top=1cm,bottom=1cm}

\definecolor{dkgreen}{rgb}{0,0.6,0}
\definecolor{gray}{rgb}{0.5,0.5,0.5}
\definecolor{mauve}{rgb}{0.58,0,0.82}

\lstset{frame=tb,
	language=Matlab,
	aboveskip=3mm,
	belowskip=3mm,
	showstringspaces=false,
	columns=flexible,
	basicstyle={\small\ttfamily},
	numbers=none,
	numberstyle=\tiny\color{gray},
	keywordstyle=\color{blue},
	commentstyle=\color{dkgreen},
	stringstyle=\color{mauve},
	breaklines=true,
	breakatwhitespace=true,
	tabsize=3
}

\begin{document}
	\maketitle
	\tableofcontents
	\section{Code And Implementation}
	\subsection{Code For Filtering}
	\lstinputlisting[breaklines]{implement.m}
	

	\newcommand{\RNum}[1]{\uppercase\expandafter{\romannumeral #1\relax}}
	\subsection{Outcome Collections}
	\newpage
	\begin{figure}[hbt!]
		\centering
		\begin{subfigure}[b]{0.45\linewidth}
			\includegraphics[width=\linewidth]{idlp0.3h.jpg}
			\caption{ideal-lp-0.3h.}
		\end{subfigure}
		\begin{subfigure}[b]{0.45\linewidth}
			\includegraphics[width=\linewidth]{idlp0.7h.jpg}
			\caption{ideal-lp-0.7h.}
		\end{subfigure}
		\caption{Ideal Lowpass}
	\end{figure}
\begin{figure}[hbt!]
	\centering
	\begin{subfigure}[b]{0.3\linewidth}
		\includegraphics[width=\linewidth]{idhp0.1h.jpg}
		\caption{ideal-hp-0.1h.}
	\end{subfigure}
	\begin{subfigure}[b]{0.3\linewidth}
		\includegraphics[width=\linewidth]{idhp0.3h.jpg}
		\caption{ideal-hp-0.3h.}
	\end{subfigure}
	\begin{subfigure}[b]{0.3\linewidth}
		\includegraphics[width=\linewidth]{idhp0.7h.jpg}
		\caption{ideal-hp-0.7h.}
	\end{subfigure}
		\caption{Ideal Highpass}
\end{figure}
\begin{figure}[hbt!]
	\centering
	\begin{subfigure}[b]{0.3\linewidth}
		\includegraphics[width=\linewidth]{btwlp0.1h1n.jpg}
		\caption{btwlp-o.1h-1n.}
	\end{subfigure}
	\begin{subfigure}[b]{0.3\linewidth}
		\includegraphics[width=\linewidth]{btwlp0.1h5n.jpg}
		\caption{btwlp-0.1h-5n.}
	\end{subfigure}
	\begin{subfigure}[b]{0.3\linewidth}
			\includegraphics[width=\linewidth]{btwlp0.1h20n.jpg}
			\caption{btwlp-0.1h-20n.}
	\end{subfigure}
		\begin{subfigure}[b]{0.3\linewidth}
		\includegraphics[width=\linewidth]{btwlp0.5h1n.jpg}
		\caption{btwlp-0.5h-1n.}
	\end{subfigure}
	\begin{subfigure}[b]{0.3\linewidth}
		\includegraphics[width=\linewidth]{btwlp0.5h5n.jpg}
		\caption{btwlp-0.5h-5n.}
	\end{subfigure}
		\begin{subfigure}[b]{0.3\linewidth}
			\includegraphics[width=\linewidth]{btwlp0.5h20n.jpg}
			\caption{btwlp-0.5h-20n.}
		\end{subfigure}
			\caption{Butterworth Lowpass}
\end{figure}
\newpage
\begin{figure}[hbt!]
	\centering
	\begin{subfigure}[b]{0.3\linewidth}
		\includegraphics[width=\linewidth]{btwhp0.1h1n.jpg}
		\caption{btwhp-o.1h-1n.}
	\end{subfigure}
	\begin{subfigure}[b]{0.3\linewidth}
		\includegraphics[width=\linewidth]{btwhp0.1h5n.jpg}
		\caption{btwhp-0.1h-5n.}
	\end{subfigure}
		\begin{subfigure}[b]{0.3\linewidth}
			\includegraphics[width=\linewidth]{btwhp0.1h20n.jpg}
			\caption{btwhp-0.1h-20n.}
	\end{subfigure}	
			\begin{subfigure}[b]{0.3\linewidth}
				\includegraphics[width=\linewidth]{btwhp0.5h1n.jpg}
				\caption{btwhp-0.5h-1n.}
			\end{subfigure}
			\begin{subfigure}[b]{0.3\linewidth}
				\includegraphics[width=\linewidth]{btwhp0.5h5n.jpg}
				\caption{btwhp-0.5h-5n.}
			\end{subfigure}
				\begin{subfigure}[b]{0.3\linewidth}
					\includegraphics[width=\linewidth]{btwhp0.5h20n.jpg}
					\caption{btwhp-0.5h-20n.}
				\end{subfigure}
					\caption{Butterworth Highpass}
	\end{figure}
	
	\begin{figure}[hbt!]
		\centering
		
				\begin{subfigure}[b]{0.3\linewidth}
					\includegraphics[width=\linewidth]{gaulp0.1h.jpg}
					\caption{gaussian-lp-0.1h.}
				\end{subfigure}
				\begin{subfigure}[b]{0.3\linewidth}
					\includegraphics[width=\linewidth]{gaulp0.3h.jpg}
					\caption{gaussian-lp-0.3h.}
				\end{subfigure}
					\begin{subfigure}[b]{0.3\linewidth}
						\includegraphics[width=\linewidth]{gaulp0.7h.jpg}
						\caption{gaussian-lp-0.7h.}
				\end{subfigure}
						\caption{Gaussian Lowpass}
	\end{figure}

\begin{figure}[hbt!]
	\centering
	
	\begin{subfigure}[b]{0.3\linewidth}
		\includegraphics[width=\linewidth]{gauhp0.1h.jpg}
		\caption{gaussian-hp-0.1h.}
	\end{subfigure}
	\begin{subfigure}[b]{0.3\linewidth}
		\includegraphics[width=\linewidth]{gauhp0.3h.jpg}
		\caption{gaussian-hp-0.3h.}
	\end{subfigure}
	\begin{subfigure}[b]{0.3\linewidth}
		\includegraphics[width=\linewidth]{gauhp0.7h.jpg}
		\caption{gaussian-hp-0.7h.}
	\end{subfigure}
	\caption{Gaussian Highpass}
\end{figure}

		
	\section{Discussion}
1.For ideal lowpass filter, when increasing the filter radius, more high frequencies will be included which makes the filtered image more clear. For highpass filter, the image will be more vague when increasing the radius. Because more low frequency energy got lost.\\
2.When increasing the order of butterworth filter, image got more vague.\\
3.When the order get big enough, the butterworth filter approximates to ideal filter. At this condition, with the same radius of filter, both two has the similar filtering effect. While, for gaussian filter, with the same radius, the image got more clear after filtered by gaussian than ideal filter for both lowpass and highpass.
\end{document}